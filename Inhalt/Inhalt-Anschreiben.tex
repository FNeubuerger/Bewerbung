%\oldSymbols % Marvosymbols

%\makeatletter
%%\@setplength{subjectaftervskip}{\baselineskip+5pt} % 29pt default %Abstand Betreff <-> Anrede
%%\@addtoplength{refvpos}{-8mm} %Textbereich 8mm nach oben
%\makeatother 
%\setlength{\footskip}{0pt} % Abstand Anlagen Seitenende anpassen (auch negativ, wie -8pt)
% Worttrennungshilfen
%	\linebreak \mbox{} \sloppy  \showhyphens{}   \hyphenation{} Wort\-trennbeispiel
\emergencystretch=20pt\tolerance=1200\hyphenpenalty=1000% Gewichte für Trennung
%\hyphenation{Louisiana}% nicht trennen
% zeilenumbrueche mit \linebreak
% Worttrennung mit Wort\-trennung erzwingen


\begin{letter}{%
Hochschule Rhein-Waal\\
Fakultät Kommunikation und Umwelt\\
Dezernat Personal\\
Marie-Curie-Straße 1\\
47533 Kleve%
}

\setkomavar{subject}[Betreff ]{Bewerbung auf die Heinz-Trox-Stiftungsprofessur für Angewandte Künstliche Intelligenz}
\opening{Sehr geehrte Berufungskommission,}

% --- Einleitung: Bezug auf Stelle + Kurzprofil ---
ich bewerbe mich auf die Heinz-Trox-Stiftungsprofessur für Angewandte Künstliche Intelligenz (Kennziffer 01/F4/26) an Ihrer Fakultät Kommunikation und Umwelt.
Die Ausschreibung spricht mich besonders an, weil sie interdisziplinäre Zusammenarbeit zwischen Informatik, Ingenieurwissenschaften und Design mit dem praxisorientierten Einsatz von KI verbindet -- genau an dieser Schnittstelle liegt meine Forschung und Lehre der letzten Jahre.

% --- Wissenschaftliche Laufbahn ---
Mein wissenschaftlicher Werdegang begann mit einem Physikstudium, in dem ich neben der Vertiefung in Astroteilchenphysik Nebenfächer in Informatik, Wirtschaftswissenschaften und Wissenschaftsphilosophie belegte.
Diese breite Grundlage prägt mein Verständnis von KI als interdisziplinäre Disziplin.
Seit Dezember 2020 promoviere ich am Promotionszentrum Angewandte Informatik zum Thema Machine Learning in der Fertigungsüberwachung. Diese Promotion steht nun vor dem Abschluss.
Im Industrieprojekt WiTraPres entwickelte ich Verfahren zur Anomalieerkennung bei Presshärteprozessen unter Einsatz von Bayesschen neuronalen Netzen, Autoencodern und synthetischer Zeitreihengenerierung.
Weitere Schwerpunkte sind Explainable AI und Wissensmanagement mit modernen Agentensystemen.

% --- Lehre + Praxis ---
Parallel zur Promotion lehre ich seit über sieben Jahre Deep Learning, Explainable AI, Maschinelles Lernen und Advanced Natural Language Processing mit LLMs an der FH Südwestfalen und betreue Bachelor- und Masterarbeiten in Kooperation mit Unternehmen.
Als wissenschaftlicher Mitarbeiter war ich zudem an der erfolgreichen Beantragung von weiteren Drittmittelprojekten beteiligt und habe Netzwerkevents zum Austausch zwischen KI-Forschung und mittelständischen Unternehmen durchgeführt. 
Ergänzend engagiere ich mich in der Open-Source-Community, um Forschungsergebnisse und Tools einer breiten Öffentlichkeit zugänglich zu machen.
Im Rahmen der zdi-Initiative habe ich an der FH Südwestfalen Veranstaltungen für Kinder und Jugendliche durchgeführt, um diese frühzeitig für technische und naturwissenschaftliche Themen zu begeistern -- eine Arbeit, die ich an den zdi-Zentren der Hochschule Rhein-Waal sehr gerne fortsetzen würde.
Im Rahmen des FH-Tandem-Programms bereite ich mich gezielt auf eine HAW-Professur vor und verbinde dabei meine industrielle Tätigkeit, bei der Infineon AG, systematisch mit der akademischen Lehre und Forschung.
Für mich liegt der Kern angewandter KI-Forschung im Wechselspiel zwischen Industrie und Wissenschaft: Die Industrie bringt konkrete, komplexe Probleme, die Forschung liefert abstrakte Lösungsansätze -- diese zusammenzubringen und Ergebnisse an realen Daten zu sehen, ist meine Motivation.

% --- Bezug Hochschule + Heinz-Trox-Stiftung ---
Die Hochschule Rhein-Waal mit ihrem Leitbild der Vielfalt, Internationalität und dem Fokus auf Praxistransfer passt zu diesem Verständnis.
Die Heinz-Trox-Stiftung mit ihrem Engagement für technologische Innovation und regionale Entwicklung am Niederrhein teilt diesen Anspruch.
Dass die Stiftungsprofessur Forschung und Lehre mit gesellschaftlichem Nutzen verbindet, entspricht meiner Überzeugung, dass technologischer Fortschritt nur durch aktiven Transfer zwischen Wissenschaft und Praxis entsteht.

% --- Vision ---
Konkret möchte ich an der Fakultät ein anwendungsorientiertes KI-Labor aufbauen, in dem Studierende aus Medieninformatik, Usability Engineering und Wirtschaftswissenschaften gemeinsam an realen Problemstellungen arbeiten -- und dabei nicht nur fachlich wachsen, sondern auch Problemlösungskompetenz und interdisziplinäre Zusammenarbeit einüben.
Mein Forschungsfokus auf Wissensmanagement mit Agentensystemen und Collective Intelligence bietet Anknüpfungspunkte für die bestehenden Studiengänge ebenso wie für Kooperationen mit regionalen Unternehmen.
Langfristig sehe ich die Möglichkeit, interdisziplinäre Drittmittelprojekte zu initiieren, die KI verantwortungsvoll in wirtschaftliche, technische und gesellschaftliche Anwendungsdomänen bringen.
Darüber hinaus bringe ich mich gern in die akademische Selbstverwaltung ein, um die Weiterentwicklung von Studiengängen und eine lebendige Hochschulkultur aktiv mitzugestalten.


%\vfill %bei weniger als 3 zeilen platz
\newline\newline\newline 
Mit freundlichen Grüßen\\
\sig%
\newline\newline\newline
\encl: Lebenslauf, Zeugnisse, Publikationsliste
%\vspace*{2mm}Anlagen %spart Platz
\end{letter}
